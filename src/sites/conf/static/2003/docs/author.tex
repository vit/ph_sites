%  This file follows LaTeX-2e specifications. 
%  LaTeX 2.09 users must replace the first line for
%  \documentstyle{physcon}  
   
\documentclass{physcon}
\begin{document}
\title{Author's Guide\\
International Conference\\
``Physics and Control'' (PhysCon 2003)
}

\author{Alexander N. Churilov\\                               
Department of Computer Science\\
St. Petersburg Marine Technical University\\
3 Lotsmanskaya Str.,
190008 St. Petersburg, Russia\\
churilov@nm.ru, http://churilov.nm.ru}
\maketitle

\begin{abstract}
This document provides instructions for 
preparing a paper for Proceedings of the 
International Conference 
``Physics and Control" (PhysCon 2003).
Conference proceedings on CD-ROM and program volume will
be distributed among the participants at the Registration Desk. 
Additionally, hard copy of the proceedings (four volumes are planned) 
can be ordered (see registration form). 
These volumes will also be distributed by IEEE CSS.

Our guidelines basically follow those adopted for IEEE CDC 
publications. In particular, we used the instructions      
written for IEEE Conferences on 
Decision and Control held in 1999--2002.
This document describes the preparation of a conference 
paper. 
\end{abstract}

\section{Paper Submission}

Authors submit their papers to the 
Organizing Committee in an electronic form.
Please follow carefully the submission process described
on the conference site {\bf http://www.physcon.ru}.
The deadline for receipt of your contribution 
is {\bf May 15, 2003}.  
Papers received after the deadline, may not be 
included in the proceedings. 
Hardcopies are not required. 
Please, apply exclusively the process pointed above, and 
do not send your manuscripts to the members
of Program or Organizing Committees by e-mail.

\section{General Specifications}

\subsection{Important Information}

\begin{enumerate}
\item You need to submit you manuscript
as PDF or Postscript file (PDF format is preferable). 
Please avoid nonstandard fonts that may cause reading 
or conversion problems. Postscript Type~1 fonts are desirable.
Any other file formats (Microsoft Word, \LaTeX, etc) are not
acceptable. For more details please see the section
``Technical information.''
The size of the file preferably should not exceed 1.5~Mb. 
(This may be achieved by avoiding complicated high-resolution 
figures.) 
Please zip your file if its size exceeds 500~Kb. 

\item When producing you PDF file please set 
A4 (210~mm by 297~mm) paper size and 600 dpi resolution.
All the descriptions of nonstandard fonts should 
be embedded as subsets into your document.
Do not generate your PDF file as image (without fonts recognition).
(See the section ``Technical information.'')
     
\item  Those participants who will give a Plenary 
or Invited Lecture can submit a full-size 
paper of twelve (12) pages, each Regular or Poster Lecture 
is allotted a paper of six (6) pages in the proceedings.  
Each manuscript must follow this rule.  Authors with 
multiple submissions must apply these guidelines to each 
manuscript separately.
Up to three additional pages will be permitted for 
a charge of 50 US Dollars/Euro for each additional page.

\item The title and authorship of your paper will 
appear in the final program, in the table of contents, 
and author index of the proceedings as they are 
shown in your manuscript submitted to 
the Organizing Committee.  
\end{enumerate}
 
\subsection{Preparation of Papers}

Your paper must be formatted in two columns.  
To ensure uniformity of appearance for the proceedings, 
the papers must conform to the following specifications. 
It is highly recommended to use \LaTeX~ in preparing and
formatting your manuscript.

The hight of the text block must be {\bf 240~mm}.
The width of each column must be {\bf 80~mm}.
The distance between the two columns of text 
must be {\bf 5~mm}. Thus, the width of the text block
is {\bf 165~mm}.
The text must be centered left-to-right 
on the page, i.e., left and right margins 
should be the same.
The distance from the bottom 
edge of the paper to the bottom of the last 
line of type on the page should be not less than {\bf 30~mm}.  
This allows room for the printer to 
insert the copyright notice on the first page 
and page numbers.

Please, do not insert page numbers or other informaton into
running headers or footers. This will be done by us.

\subsection{Required Font Sizes}

For the main body of the text the {\bf 10} point size roman font
should be used. 
Some technical formatting programs print 
mathematical formulas in italic type, with subscripts and 
superscripts in a slightly smaller font size. This is acceptable.

\subsection{Title} The title should be centered across the top 
of the first page. The {\bf 18} point size bold font is recommended.

\subsection{Authors' Names and Addresses} The authors' names and
addresses must be centered below the title. The {\bf 12} point 
size roman font is recommended. Please, 
include your e-mail address. 
For multiauthored papers names and addresses can be placed
either side by side, or one under another. Please do not
put addresses in footnotes.

\subsection{Paragraphs}
Do not indent the initial lines of paragraphs. 
Leave a line clear between paragraphs.

\subsection{References} References should be numbered
consecutively throughout the paper and be listed at the end
of the paper with the main heading {\bf References}. 
When citing references in the text, type the
corresponding number in square brackets \cite{1}. 
References should be complete and in standard IEEE style.
See the section {\bf References} for examples of listing
different kinds of publications \cite{1}--\cite{3}.

\subsection{Equations}
Equations have to be numbered consecutively with the number
in parenthesis, flush to the right. E.g., see 
the equation (\ref{E1})
\begin{equation}                        \label{E1}                 
\sigma(t)=\sigma_0(t)-\int_0^T \gamma(t-s)u(s)\,ds.
\end{equation}

\subsection{Theorems, Lemmas, Remarks, Definitions and so on}
The body of a theorem, or a similar statement, can be typed 
either in roman, or in italic font. When using \LaTeX~employ the 
\verb|\newtheorem| command. E.g.,
\begin{lemma}
Body of the lemma. Body of the lemma. Body of the lemma.
Body of the lemma. Body of the lemma. Body of the lemma.
Body of the lemma. Body of the lemma. Body of the lemma.
\end{lemma}

For proofs the following style is recommended:

\begin{proof}
Body of the proof. Body of the proof. Body of the proof.
Body of the proof. Body of the proof. Body of the proof.
Body of the proof. Body of the proof. Body of the proof.
\end{proof}
When using LaTeX, the {\bf proof} environment, defined
in the style file {\bf physcon.sty} can be applied.

\subsection{Illustrations}

Please avoid halftone illustrations and photographs.
The papers will be printed in black and white. 
Colored figures are acceptable for CD~ROM
publications, but please make sure that they are reproduced 
clearly in gray scale and the text does not refer to colors.

Figures should be inserted near their citation or at the end of the 
manuscript, after References.  Large figures may extend 
over two columns if necessary.  Make sure that you number 
and include a caption 
for each figure. See Fig.~\ref{Fig1} and Fig.~\ref{Fig2}
for examples.
\begin{figure}
\begin{center}
\begin{picture}(220,60)
\put(20,40){\circle{15}}             
\put(23,28){{\small $-$}}
\put(15,35){\line(1,1){10}}         
\put(25,35){\line(-1,1){10}}        
\put(28,40){\vector(1,0){36}}       
\put(34,45){$\sigma(t)$}
\put(64,30){\framebox(40,20){PM}}  
\put(104,40){\vector(1,0){36}}      
\put(109,45){$f(t)$}
\put(140,30){\framebox(40,20){CLP}}
\put(180,40){\line(1,0){32}}       
\put(212,40){\line(0,-1){30}}      
\put(212,10){\line(-1,0){192}}     
\put(20,10){\vector(0,1){23}}      
\end{picture}
\end{center}
\caption{One-column sample picture}\label{Fig1}
\end{figure}
\begin{figure*}
\begin{center}
\begin{picture}(300,100)
\put(50,80){\circle{15}}             
\put(35,85){{\small $+$}}
\put(53,68){{\small $-$}}
\put(45,75){\line(1,1){10}}         
\put(55,75){\line(-1,1){10}}        
\put(8,80){\vector(1,0){34}}        
\put(8,85){$\psi(t)$}
\put(58,80){\vector(1,0){36}}       
\put(64,85){$\sigma(t)$}
\put(94,70){\framebox(40,20){PWM}}
\put(134,80){\vector(1,0){36}}
\put(152,80){\circle*{2}}
\put(138,85){$f(t)$}
\put(178,80){\circle{15}}
\put(163,85){{\small $+$}}
\put(181,68){{\small $-$}}
\put(173,75){\line(1,1){10}}         
\put(183,75){\line(-1,1){10}}        
\put(185.5,80){\vector(1,0){30}}
\put(216,70){\framebox(40,20){$W(s)$}}
\put(271,85){$\xi(t)$}
\put(256,80){\line(1,0){42}}
\put(298,80){\line(0,-1){70}}
\put(298,10){\line(-1,0){248}}
\put(50,10){\vector(0,1){62}}

\put(216,30){\framebox(40,20){$R(s)$}}
\put(276,80){\circle*{2}}
\put(276,80){\line(0,-1){40}}
\put(276,40){\vector(-1,0){20}}
\put(216,40){\line(-1,0){38}}
\put(190,45){$v(t)$}
\put(178,40){\vector(0,1){10}}
\put(178,50){\circle*{2}}
\put(178,63){\vector(0,1){10}}
\put(178,63){\circle*{2}}
\put(152,80){\line(0,-1){26}}
\put(152,54){\vector(1,0){20.5}}
\thicklines
\put(178,50){\line(-1,1){10}}
\end{picture}
\end{center}
\caption{Two-columns sample picture}\label{Fig2}
\end{figure*}

\subsection{Tables}
Table heading, including the number, is to be typed
above the table. See, e.g., 
\begin{table}[htb]
\begin{center}
\caption{Caption text}
\begin{tabular}{lll}
\hline
Title 1 & Title 2 & Title 3\\
\hline
Row 1, Col 1 & Row 1, Col 2 & Row 1, Col 3\\
Row 2, Col 1 & Row 2, Col 2 & Row 2, Col 3\\
Row 3, Col 1 & Row 3, Col 2 & Row 3, Col 3\\
\hline
\end{tabular}
\end{center}
\end{table}

\subsection{Footnotes}
Please, do not use footnotes at all. 
Authors' addresses have to be
placed after the authors' names below the title of the paper. 
Information about any financial support should be placed 
in the subsection {\bf Acknowledgements} at the end of your paper. 

\subsection{Page Numbers, Session Number, 
Copyright Information and Conference Identifications}

Please, do not number pages. All the information listed above
will be inserted later by the proceedings printer.

\section{Headings}

Main headings are to be column centered in a bold font. 
They may be numbered, if so desired. 

\subsection{Subheadings}
Subheadings should be in a bold font. They should start at 
the left-hand margin on a separate line. 

\subsubsection{Sub-subheadings} They are to be in a bold font,
should be indented and run in at the beginning of the paragraph.

\section{Hints for \LaTeX~Users}

It is highly recommended that your paper be
prepared using \LaTeX~document preparation system.
We provide \LaTeX~2\raisebox{-2pt}{$\varepsilon$}
users with a class file {\bf physcon.cls} and  
\LaTeX~2.09 users with a style file {\bf physcon.sty}.  
Both files are available from PhysCon home page
{\bf http://www.physcon.ru}.
The \LaTeX~file, containing this instruction, is partly
given in Appendix and can serve as a specimen. 

\section{Technical Information}

\subsection{How to Produce a Postscript File}

A straightforward way to obtain a Postscript file from a file
of any format is to print it with a Postscript driver.
If you use a Postscript printer, you also have such driver
installed. Otherwise, you may install a driver for a Postscript
printer (e.g., for Apple Laser Writer) from standard distributives.
Details depend on your platform,
e.g., for Windows use Control Panel/Add Printer.
Configure the driver to download all fonts (this may be automatic), 
and if available configure the driver for maximum postscript portability.
Do not use internal printer fonts. 
Then print your file employing ``Print to File''
option.

The most common way is to produce a Postscript file from \TeX\
is using the dvips.exe utility (it is included in most of the modern
\TeX\ distributions and also available from the website
{\tt http://www.radicaleye.com/dvips}).
To obtain later better readible PDF fonts, users 
should apply this utility with the Ppdf option, e.g.:\\ 
{\tt dvips -Ppdf filename.dvi}

\subsection{Requirements to a PDF File}

The conference CD ROM will contain a collection of files
in the PDF file format. Such files can be viewed on screen
or printered using the free software named Acrobat Reader,
which is developed by Adobe Systems and has versions for the 
most popular platforms (Windows, Mac, and Unix).

Files with conference papers should be fully portable.
It means that they should be properly viewed and printered
at any computer in any country. It can be achieved by embedding
fonts into PDF documents. 
You make check the fonts you used with the 
Acrobat Reader menus:
{\tt  File/Document Info/Fonts}.    

\subsection{How to Produce a PDF File from Postscript}

To convert a Postscript file to a PDF file you need a program
called Adobe Acrobat Distiller, which is a part of Adobe Acrobat
package. Unlike Acrobat Reader, this program is not free,
so, if you do not have it, we will execute the conversion by
ourselves.

A Postript file or a Microsoft Word file can serve as an input
for this program.

Within Distiller's ``Job Options'' dialog 
(select using menu item {\tt Distiller/Job Options}), 
configure the following:

On the ``Font Embedding'' tab turn ON the ``Embed all fonts'' control.
 
Turn ON the ``Subset fonts below'' control and type in the value 100 
(i.e. any font below 100\% usage will be stored in the Acrobat file 
as a subset of the original font.) 
This subset operation is necessary to protect the copyright of 
the font creator. 

On the ``Compression'' tab, turn ON the ``Compress Text and Line Art'' 
control. 

Turn OFF the ``Downsample'' control for color, grayscale and monochrome 
images. (Image size is permitted to be large when the document 
is stored on CDROM.) 

Turn ON ``Manual Compression'' for color images and set to JPEG High.
 
Turn ON ``Manual Compression'' for grayscale images and set to JPEG High.
 
Turn ON ``Manual Compression'' for monochrome images and set to 
CCITT Group 4. 


{\em Important note.} Another program that is commonly used for
Postscript--PDF conversion is Acrobat Writer. However, this
program does not properly embed fonts into a document. 
That is why we ask you {\em not to apply Acrobat Writer}.

\subsection{How to Produce a PDF File from \TeX}

There are several ways to produce a PDF document 
immediately from \TeX.
(A comprehensive treatment of the problem can be found in
Chapter~2 \cite{1}.) Here we describe two of them.

(i) You may use the dvipdfm.exe utility developed by
M.A.~Wicks. Its Windows version is included in the latest
MiK\TeX\ distributions 
{\tt http://www.miktex.org} or \TeX Live package. 
Its LINUX version is
available from the website\\ 
{\tt  http://gaspra.kettering.edu/dvipdfm}.\\
This utility produces excellent quality documents immediately
from a dvi-file, it uses partial fonts embedding (i.e., embeds
only those characters that are needed in the document).  
Some efforts are required when including images: only the images
of JPG, PNG, and PDF formats are admissible, some special commands
should be used for their inclusion.
Please, take into account that {\tt dvipdf} uses Letter paper format
by default, so A4 format should be indicated explicitely in
the command line, e.g.:\\
{\tt dvipdfm -p a4 filename.dvi}.

(ii) The second way is to use PDF\TeX; it is also contained in 
the last distributions of MiK\TeX\ and \TeX Live.
The PDF\TeX\ programs produce pdf-files instead of dvi-files.
You may just use {\bf pdflatex.exe} program, e.g.:\\
{\tt   pdflatex filename.tex}.

\section{IEEE Copyright Form}

Complete the IEEE Copyright form available at our cite. 
No paper can be published without this completed form. 
The completed and signed form should be mailed 
to us to arrive by 15 May 2003. 

\section{Conclusions}
                                  
Please make an extra effort to follow these guidelines 
as the quality of the publications depends on you. 
Thank you for your cooperation and contribution. 
We look forward to seeing you at our conference in 
St.~Petersburg.

\subsection*{Acknowledgements}

This subsection can be used to acknowledge any kind 
of support to your paper (e.g., you can mention your grants).    

\begin{thebibliography}{55}
\bibitem{1} M.~Goosens and S.~Rahtz,
    {\it The \LaTeX Web Companion},
    Addison--Wesley, 1999.
    (A Russian translation is published by Mir,  
    2001.)
\bibitem{2} A.L.~Fradkov, ``Synthesis of an adaptive system
    of linear plant stabilization,'' {\it Avtomat. i Telemekh.},
    no.~12, pp.~96--103, 1974 (Russian).
\bibitem{3} H.~Nijmeijer, I.I.~Blekhman, A.L.~Fradkov,
    and A.Yu.~Pogromsky, 
    ``Self-synchronization and controlled synchronization,"
    {\it Proceedings of the 1st International Conference on 
    Control of Oscillations and Chaos 
    COC'97}, St.~Petersburg, Russia, 
    August 27--29 1997, vol.~1, pp.~36--41.
\end{thebibliography}

\section*{Appendix}
Here you find a shortened version of these
instructions written in \LaTeX.
  
\begin{verbatim}
%  This text follows LaTeX-2e  
%  specifications. LaTeX 2.09 users 
%  have to replace its first line for
%  \documentstyle{physcon}  
%
\documentclass{physcon}  
\begin{document}

\title{Author's Guide\\
International Conference\\
``Physics and Control'' 
(PhysCon 2003)
}

\author{Alexander N. Churilov\\                               
Department of Computer Science\\
St. Petersburg Marine Technical 
University\\
3 Lotsmanskaya Str.,
190008 St.~Petersburg, Russia\\
churilov@nm.ru, 
http://churilov.nm.ru}
\maketitle

\begin{abstract}
...........................................
\end{abstract}

\section{Manuscript Submission}

Authors submit their manuscripts to the 
Organizing Committee.
...........................................  

\section{General Specifications}
...........................................  

\subsection{References} References should 
be numbered consecutively throughout the 
paper and be listed at the end of the paper 
with the main heading {\bf References}. 
When citing references in the text, type 
the corresponding number in square brackets 
\cite{1}. References should be 
complete and in standard IEEE style. 
See the section {\bf References} for 
examples of listing different kinds of 
publications \cite{1}--\cite{3}.

\subsection{Equations}
Equations have to be numbered consecutively 
with the number in parenthesis, flush to 
the right. 
E.g., see the equation (\ref{E1})
\begin{equation}                 \label{E1}                 
\sigma(t)=\sigma_0(t)-\int_0^T 
                       \gamma(t-s)u(s)\,ds.
\end{equation}

\subsection{Theorems, Lemmas, Remarks, 
Definitions and so on}
The body of a theorem, or a similar 
statement, can be typed either in roman, or 
in italic font. 
When using \LaTeX~employ 
the \verb|\newtheorem| command. E.g.,
\begin{lemma}
Body of the lemma. Body of the lemma. 
...........................................
\end{lemma}

For proofs the following style is 
recommended:

\begin{proof}
Body of the proof. Body of the proof. 
...........................................
\end{proof}
...........................................

\subsection{Illustrations}
...........................................
 
Make sure that you number and include 
a caption for each figure. See 
Fig.~\ref{Fig1} and Fig.~\ref{Fig2} 
for examples.
\begin{figure}
\begin{center}
\begin{picture}(220,60)
...........................................
\end{picture}
\end{center}
\caption{One-column sample 
                      picture}\label{Fig1}
\end{figure}
\begin{figure*}
\begin{center}
\begin{picture}(300,100)
...........................................
\end{picture}
\end{center}
\caption{Two-columns sample 
                      picture}\label{Fig2}
\end{figure*}

\subsection{Tables}
Table heading, including the number, is to 
be typed above the table. See, e.g., 
\begin{table}[htb]
\begin{center}
\caption{Caption text}
\begin{tabular}{lll}\hline
Title 1 & Title 2 & Title 3\\ \hline
Row1, Col1 & Row1, Col2 & Row1, Col3\\
Row2, Col1 & Row2, Col2 & Row2, Col3\\
Row3, Col1 & Row3, Col2 & Row3, Col3\\
\hline
\end{tabular}
\end{center}
\end{table}
...........................................

\section{Headings}

Main headings are to be column centered 
in a bold font. 
They may be numbered, if so desired. 

\subsection{Subheadings}
Subheadings should be in a bold font. They 
should start at the left-hand margin on 
a separate line. 

\subsubsection{Sub-subheadings} They are 
to be in a bold font, should be indented 
and run in at the beginning of 
the paragraph.

...........................................

\subsection*{Acknowledgements}

This subsection can be used to acknowledge 
any kind of support to your paper 
(e.g., you can mention your grants).    

\begin{thebibliography}{55}
\bibitem{1} M.~Goosens and S.~Rahtz,
    {\it The \LaTeX Web Companion},
    Addison--Wesley, 1999.
    (A Russian translation is published 
     by Mir, 2001.)
\bibitem{2} A.L.~Fradkov, ``Synthesis of 
an adaptive system of linear plant 
stabilization,'' 
{\it Avtomat. i Telemekh.},
no.~12, pp.~96--103, 1974 (Russian).
...........................................
\end{thebibliography}
\end{document}
\end{verbatim}
\end{document}



